% Options for packages loaded elsewhere
\PassOptionsToPackage{unicode}{hyperref}
\PassOptionsToPackage{hyphens}{url}
%
\documentclass[
  11pt,
  ignorenonframetext,
]{beamer}
\usepackage{pgfpages}
\setbeamertemplate{caption}[numbered]
\setbeamertemplate{caption label separator}{: }
\setbeamercolor{caption name}{fg=normal text.fg}
\beamertemplatenavigationsymbolsempty
% Prevent slide breaks in the middle of a paragraph
\widowpenalties 1 10000
\raggedbottom
\setbeamertemplate{part page}{
  \centering
  \begin{beamercolorbox}[sep=16pt,center]{part title}
    \usebeamerfont{part title}\insertpart\par
  \end{beamercolorbox}
}
\setbeamertemplate{section page}{
  \centering
  \begin{beamercolorbox}[sep=12pt,center]{part title}
    \usebeamerfont{section title}\insertsection\par
  \end{beamercolorbox}
}
\setbeamertemplate{subsection page}{
  \centering
  \begin{beamercolorbox}[sep=8pt,center]{part title}
    \usebeamerfont{subsection title}\insertsubsection\par
  \end{beamercolorbox}
}
\AtBeginPart{
  \frame{\partpage}
}
\AtBeginSection{
  \ifbibliography
  \else
    \frame{\sectionpage}
  \fi
}
\AtBeginSubsection{
  \frame{\subsectionpage}
}
\usepackage{amsmath,amssymb}
\usepackage{lmodern}
\usepackage{iftex}
\ifPDFTeX
  \usepackage[T1]{fontenc}
  \usepackage[utf8]{inputenc}
  \usepackage{textcomp} % provide euro and other symbols
\else % if luatex or xetex
  \usepackage{unicode-math}
  \defaultfontfeatures{Scale=MatchLowercase}
  \defaultfontfeatures[\rmfamily]{Ligatures=TeX,Scale=1}
\fi
\usetheme[]{metropolis}
% Use upquote if available, for straight quotes in verbatim environments
\IfFileExists{upquote.sty}{\usepackage{upquote}}{}
\IfFileExists{microtype.sty}{% use microtype if available
  \usepackage[]{microtype}
  \UseMicrotypeSet[protrusion]{basicmath} % disable protrusion for tt fonts
}{}
\makeatletter
\@ifundefined{KOMAClassName}{% if non-KOMA class
  \IfFileExists{parskip.sty}{%
    \usepackage{parskip}
  }{% else
    \setlength{\parindent}{0pt}
    \setlength{\parskip}{6pt plus 2pt minus 1pt}}
}{% if KOMA class
  \KOMAoptions{parskip=half}}
\makeatother
\usepackage{xcolor}
\newif\ifbibliography
\setlength{\emergencystretch}{3em} % prevent overfull lines
\providecommand{\tightlist}{%
  \setlength{\itemsep}{0pt}\setlength{\parskip}{0pt}}
\setcounter{secnumdepth}{-\maxdimen} % remove section numbering
\ifLuaTeX
  \usepackage{selnolig}  % disable illegal ligatures
\fi
\IfFileExists{bookmark.sty}{\usepackage{bookmark}}{\usepackage{hyperref}}
\IfFileExists{xurl.sty}{\usepackage{xurl}}{} % add URL line breaks if available
\urlstyle{same} % disable monospaced font for URLs
\hypersetup{
  pdftitle={Construcción de un diagrama de flujo},
  pdfauthor={Gerardo Martín},
  hidelinks,
  pdfcreator={LaTeX via pandoc}}

\title{Construcción de un diagrama de flujo}
\author{Gerardo Martín}
\date{2022-06-29}

\begin{document}
\frame{\titlepage}

\begin{frame}{Razones}
\protect\hypertarget{razones}{}
\begin{itemize}
\item
  Identificar necesidades

  \begin{itemize}
  \tightlist
  \item
    Materiales
  \item
    Espacio físico
  \item
    Fechas de inicio
  \item
    Fechas de toma de muestra
  \item
    Fechas de fin
  \item
    Identificación de estrategia de análisis
  \item
    Estructuración de base de datos
  \end{itemize}
\end{itemize}
\end{frame}

\begin{frame}{Técnicas}
\protect\hypertarget{tuxe9cnicas}{}
\begin{enumerate}
\tightlist
\item
  Determinar objetivo, hipótesis o pregunta de investigación
\item
  Identificar factores que se pueden manipular
\item
  Identificar maneras de manipular factores experimentalmente ó
  geográficamente (estudios observacionales)
\item
  Determinar necesidades de espacio
\item
  Hacer lista de materiales
\item
  Hacer diagrama de diseño
\end{enumerate}
\end{frame}

\hypertarget{ejemplo}{%
\section{Ejemplo}\label{ejemplo}}

\begin{frame}{El experimento}
\protect\hypertarget{el-experimento}{}
\href{https://www.scielo.org.mx/scielo.php?script=sci_arttext\&pid=S0187-57792022000100110\&lang=es}{Influencia
de tres regímenes de riego}

\href{Figuras-disenos/diagrama.odg}{Diagrama de flujo}
\end{frame}

\hypertarget{el-proyecto-de-anuxe1lisis}{%
\section{El proyecto de análisis}\label{el-proyecto-de-anuxe1lisis}}

\begin{frame}{¿Qué variable(s) de respuesta habrá?}
\protect\hypertarget{quuxe9-variables-de-respuesta-habruxe1}{}
En cosecha:

\begin{enumerate}
\tightlist
\item
  Altura de la planta
\item
  Número de espigas
\end{enumerate}

Depués de cosecha:

\begin{enumerate}
\tightlist
\item
  Tamaño de espiga
\item
  Número de granos en la espiga
\item
  Color de los granos
\item
  Peso de 1000 granos
\item
  Rendimiento
\item
  Cantidad de proteína y ceniza
\end{enumerate}
\end{frame}

\begin{frame}{¿Qué variable(s) independientes habrá?}
\protect\hypertarget{quuxe9-variables-independientes-habruxe1}{}
\begin{enumerate}
\item
  Tres regímenes de humedad del suelo, con valores:

  \begin{itemize}
  \tightlist
  \item
    Lámina de riego de 39.3 cm
  \item
    Control con lámina de 42.5 cm
  \item
    Lámina de riego de 44.8 cm
  \end{itemize}
\end{enumerate}
\end{frame}

\begin{frame}{Modelo de análisis}
\protect\hypertarget{modelo-de-anuxe1lisis}{}
Comparación de medias:

\begin{enumerate}
\tightlist
\item
  ANOVA ó ANODE
\item
  Regresión lineal
\end{enumerate}

Ejemplo ANOVA:

\[ Peso(Regimen) =  \alpha + \beta_{Regimen} \]

\begin{itemize}
\tightlist
\item
  \(\alpha =\) Promedio global de \(Peso\)
\item
  \(\beta_{Regímen} =\) Diferencia entre \(\alpha\) y peso promedio de
  cada tratamiento
\end{itemize}
\end{frame}

\begin{frame}{Modelo de análisis}
\protect\hypertarget{modelo-de-anuxe1lisis-1}{}
Ejemplo de regresión lineal:

\[ Peso(Lamina) = \alpha + \beta \times Lamina \]

\begin{itemize}
\tightlist
\item
  \(\alpha =\) Intercepto, valor de \(Peso\) cuando \(Lamina = 0\)
\item
  \(\beta =\) Pendiente, cuańto cambia el peso cuando \(Lamina\) aumenta
  en 1 unidad
\end{itemize}
\end{frame}

\begin{frame}{El análisis multivariado}
\protect\hypertarget{el-anuxe1lisis-multivariado}{}
\begin{itemize}
\item
  Los análisis anteriores son univariados

  \begin{itemize}
  \tightlist
  \item
    Sólo hay una variable de repuesta
  \end{itemize}
\item
  ¿Qué hacemos si hay dos ó más variables, como

  \begin{itemize}
  \tightlist
  \item
    Tamaño de espiga, Número de granos, peso de 1000 granos, etc.?
  \item
    Analizar la relación entre todas las variables de respuesta con
    estadística multivariada.
  \item
    Por ejemplo, ¿hay patrones que corresponden con cada uno de los
    tratamientos?
  \item
    Comenzaremos por visualizar los datos y después hacer el análisis
    multivariado
  \end{itemize}
\end{itemize}
\end{frame}

\end{document}
